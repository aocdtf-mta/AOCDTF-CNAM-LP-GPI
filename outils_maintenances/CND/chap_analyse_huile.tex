%--------------------------------------
%CANEVAS
%--------------------------------------

%utiliser les environnement \begin{comment} \end{comment} pour mettre en commentaire le préambule une fois la programmation appelée dans le document maître (!ne pas oublier de mettre en commentaire \end{document}!)

\begin{comment}

\documentclass[a4paper, 11pt, twoside, fleqn]{memoir}

\usepackage{AOCDTF-cnam}

\marqueurchapitre

%lien d'édition des figures Tikz sur le site mathcha.io (rajouter le lien d'une modification effectuée sur la figure tikz avec le nom du modificateur car il n'y a qu'un lien par compte)

%lien mathcha Nom Prénom : 

%--------------------------------------
%corps du document
%--------------------------------------

\begin{document} %corps du document
	\openleft %début de chapitre à gauche

\end{comment}

\chapter{Analyse d'huiles}
\ChapFrame

\section{Les huiles industrielles}


\subsection{Composition}

\subsubsection{Huiles de base}

\subsubsection{Additifs}


\subsection{Types d'huiles}

\subsection{Caractéristiques générales}

\subsubsection{Viscosité}

\section{Dégradations des huiles}


\subsection{Dégradation}

\subsection{Contamination}

\subsection{Effets}


\section{Méthodes d'analyse et procédés de contrôle}


\section{Maintenance}




%\end{document}


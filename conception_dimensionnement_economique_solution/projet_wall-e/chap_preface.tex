%--------------------------------------
%CANEVAS
%--------------------------------------

%utiliser les environnement \begin{comment} \end{comment} pour mettre en commentaire le préambule une fois la programmation appelée dans le document maître (!ne pas oublier de mettre en commentaire \end{document}!)

\begin{comment}

\documentclass[a4paper, 11pt, twoside, fleqn]{memoir}

\usepackage{AOCDTF-cnam}

\marqueurchapitre
\decoupagechapitre{1} %juste pour éviter les erreurs lors de la compilation des sous-programmations (passera en commentaire)


%--------------------------------------
%corps du document
%--------------------------------------

\begin{document} %corps du document
	\openleft %début de chapitre à gauche

\end{comment}

\chapter{Préface}

Ce dossier sera structuré en deux parties, la première abordant le cadrage général de la réalisation d'un projet technique en milieu professionnel dans le cadre de l'unité d'enseignement \og UE 8 : conception et dimensionnement technique et économique des solutions \fg{} octroyée en Licence GPI.\\ La deuxième compile l'ensemble des documents nécessaires à l'aboutissement du projet Wall-E et est structuré suivant un découpage projet normé (identification, étude, réalisation, évaluation).\\
Il consiste en la conception et réalisation d'un robot \og bio-inspiré \fg{} à destination des présentoirs des journées portes ouvertes des maisons de Compagnons du Devoir.\\

Le premier but de ce projet est de présenter une réalisation attractive aboutissant à la combinaison de diverses compétences liées au métiers de l'industrie et des compétences en gestion acquises dans le cadre de la formation Licence Professionnelle Gestion de Production Industrielle. Son autre but est de valider scolairement l'UE 8.\\

Ce projet Wall-E présente la particularité de comporter un \og projet dans le projet \fg{}, \og Fédération Fab Lab \fg{} (FFL), qui va prendre forme de par les financements nécessaires au projet Wall-E. Ces financements sont débloqués par le CFA des Compagnons de Tours Nord \og en échange \fg{} de l'octroi de cours du soir aux jeunes suivant les autres formations du CFA.\\
Ces cours du soir aborderont les sujets des nouvelles technologies de production (imprimante 3D, graveuse numérique\ldots) et de leurs rôles grandissant dans leurs métiers respectifs, en partenariat avec le Fab Lab de Tours donc. FFL reprendra donc le même découpage que pour Wall-E mais reste inclus dans le présent dossier.

%\end{document}


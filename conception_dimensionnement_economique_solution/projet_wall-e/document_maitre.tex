%--------------------------------------
%appel de la classe de document et de ses options
%--------------------------------------

\documentclass[a4paper, 11pt, twoside, fleqn]{memoir}

\usepackage{AOCDTF-cnam}

\addbibresource{../../bibliographies/management.bib}

%--------------------------------------
%données du document
%--------------------------------------

\corpsdemetier{Métiers des Technologies Associées}
\acronymemetier{MTA}

\formation{Licence Profesionnelle Gestion de Production Industrielle}
\matiere{Conception, dimensionnement technique et économique des solutions}
\cours{Réalisation d'un projet technique en milieu professionnel}

\auteura{LP GPI}{Promo 2020-2021}
\acronymemetierauteura{MTA}
\auteurb{}{}
\acronymemetierauteurb{nul}
\auteurc{}{}
\acronymemetierauteurc{nul}
\auteurd{}{}
\acronymemetierauteurd{nul}

\decoupagechapitre{5} %espacement des marqueurs entre les différents chapitres (à régler en fin de rédaction) (5cm vaut un espacement équivalement à la hauteur du marqueur, une page ne peut en contenir que 5 avec cet espacement-la mais il est le plus équilibré)

%--------------------------------------
%corps du document
%--------------------------------------

\begin{document} %corps du document

%--------------------------------------
%préface, page de couverture, table des matières...
%--------------------------------------

\frontmatter
	
\Framefalse %défini la booléenne Frame comme faux
	
	%--------------------------------------
	%page de couverture et de titre
	%--------------------------------------

\pagetitre
\marqueurchapitre

	\pagestyle{plain} %style de page avec en-tête et pied-de-page
	\pagenumbering{roman}
	\openany
	
	%--------------------------------------
	%listes de contenus
	%--------------------------------------
	
	{\hypersetup{linkcolor=black}\tableofcontents*} %table des matières en noir
	%\newpage
	%{\hypersetup{linkcolor=black}\listoftables*} %liste des tableaux en noir
	%\newpage
	%{\hypersetup{linkcolor=black}\listoffigures*} %liste des figures en noir
	%{\hypersetup{linkcolor=black}\listtheoremname\listtheorems{formule}} %liste des équations en noir
	%{\hypersetup{linkcolor=black}\listdefinitionname\listtheorems{definition}} %liste des définitions en noir
	%{\hypersetup{linkcolor=black}\listexemplename\listtheorems{exemple}} %liste des définitions en noir

	\openright %début de chapitre à "droite" mais comme demarrage de la numérotation inversé avec la page de titre, ça décale l'ouverture des chapitre à gauche

	%--------------------------------------
	%chapitre d'introduction
	%--------------------------------------

	%--------------------------------------
%CANEVAS
%--------------------------------------

%utiliser les environnement \begin{comment} \end{comment} pour mettre en commentaire le préambule une fois la programmation appelée dans le document maître (!ne pas oublier de mettre en commentaire \end{document}!)

\begin{comment}

\documentclass[a4paper, 11pt, twoside, fleqn]{memoir}

\usepackage{AOCDTF}

\input{marqueur_chapitre_couleur} %à personnaliser selon le nombre de chapitre dans le cours

%--------------------------------------
%corps du document
%--------------------------------------

\begin{document} %corps du document
	\openleft %début de chapitre à gauche

\end{comment}

	\chapter{Préface}
	\lipsum[1-7]

%\end{document}


		
%--------------------------------------
%corps de texte, annexes
%--------------------------------------

\mainmatter

\Frametrue %défini la booléenne Frame comme vrai -> marqueurs de chapitre

	%--------------------------------------
	%inclusion des chapitres
	%--------------------------------------
	
	\part{Projet global}
	
	%--------------------------------------
%CANEVAS
%--------------------------------------

%utiliser les environnement \begin{comment} \end{comment} pour mettre en commentaire le préambule une fois la programmation appelée dans le document maître (!ne pas oublier de mettre en commentaire \end{document}!)

%\begin{comment}

\documentclass[a4paper, 11pt, twoside, fleqn]{memoir}

\usepackage{AOCDTF-cnam}

\marqueurchapitre
\decoupagechapitre{1} %juste pour éviter les erreurs lors de la compilation des sous-programmations (passera en commentaire)


%--------------------------------------
%corps du document
%--------------------------------------

\begin{document} %corps du document
	\openleft %début de chapitre à gauche

%\end{comment}

\chapter{Identification}
\ChapFrame

\section{\'Etude préalable}

\section{Dossier de base technique}

\section{Cahier de charge client}


\end{document}






	%--------------------------------------
	%style des annexes
	%--------------------------------------

	\Framefalse %défini la booléenne Frame comme false -> pas de marqueurs de chapitre
	\appendix %appel des annexes
	\appendixpage

	%--------------------------------------
	%inclusion des chapitres
	%--------------------------------------


%--------------------------------------
%conclusion, bibliographie
%--------------------------------------

\backmatter

	%--------------------------------------
	%inclusion des chapitres
	%--------------------------------------


	%--------------------------------------
	%bibliographie
	%--------------------------------------

	\nocite{AuteurAnnee, Organisme:Numeronorme-Annee, Site:abreviation}
	
	\printbibliography %ajout des références bibliographiques
		
\end{document}

